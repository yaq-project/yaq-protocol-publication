% document
\documentclass[11pt, full]{article}
\usepackage[letterpaper, portrait, margin=0.75in]{geometry}
\usepackage{setspace}
\usepackage{color}

\RequirePackage{etoolbox}
\RequirePackage{pmboxdraw}
\RequirePackage[section]{minted}
\usemintedstyle{default}
\setminted[python]{linenos=false, frame=none}
\usepackage{lscape}

\definecolor{bg}{rgb}{0.95, 0.95, 0.95}

\RequirePackage[many]{tcolorbox}
\tcbuselibrary{minted}

% text
\usepackage[utf8]{inputenc}
\setlength\parindent{0pt}
\setlength{\parskip}{1em}
\usepackage{enumitem}
\renewcommand{\familydefault}{\sfdefault}
\newcommand{\RomanNumeral}[1]{\textrm{\uppercase\expandafter{\romannumeral #1\relax}}}

\newcommand\yaq{\texttt{yaq}}
\newcommand\yaqccmds{\texttt{yaqc-cmds}}

% math
\usepackage{amssymb}
\usepackage{amsmath}
\usepackage[cm]{sfmath}
\usepackage{commath}
\usepackage{multirow}
\DeclareMathAlphabet{\mathpzc}{OT1}{pzc}{m}{it}

% graphics
\usepackage{graphics}
\usepackage{graphicx}
\usepackage{epsfig}
\usepackage{epstopdf}
\usepackage{xpatch}
\graphicspath{{./figures/}}

% "S" prefix
\renewcommand{\theequation}{S\arabic{equation}}
\renewcommand{\thefigure}{S\arabic{figure}}
\renewcommand{\thetable}{S\arabic{table}}

% each section begins new page
\let\stdsection\section
\renewcommand\section{\clearpage\stdsection}

% force all floats to center (see https://tex.stackexchange.com/a/53383)
\makeatletter
\g@addto@macro\@floatboxreset{\centering}
\makeatother

% bibliography
\usepackage[backend=bibtex, natbib=true, url=false, sorting=none, maxbibnames=99]{biblatex}
\bibliography{references}

% hyperref
\usepackage[colorlinks=true, linkcolor=black, urlcolor=blue, citecolor=black, anchorcolor=black]
  {hyperref}
\usepackage[all]{hypcap}  % helps hyperref work properly


\NewTCBListing[number within=chapter, auto counter]{codefragment}{mo}{%
  boxrule=1pt,
  arc=0pt,
  shadow=false,
  boxsep=1ex, top=0pt, left=0pt, right=0pt, bottom=0pt,
  listing outside comment,
  righthand width=0pt,
  sidebyside gap=0pt,
  minted language=#1,
}

\BeforeBeginEnvironment{codefragment}{\begin{singlespace}\stepcounter{equation}}
\AfterEndEnvironment{codefragment}{\end{singlespace}}

\newmintinline[bash]{bash}{bgcolor=bg}
\newmintinline[python]{python}{bgcolor=bg}

\begin{document}
\pagenumbering{gobble}

\begin{center}
  \LARGE

  \textit{Supporting Information} \\
  The yaq Project: \\
  Standardized Software Enabling Flexible Instrumentation

  \normalsize

  \textit{Kyle F. Sunden, Daniel D. Kohler, Kent A. Meyer, Peter L. Cruz Parrilla, \\
    John C. Wright, and Blaise J. Thompson*}

  Department of Chemistry, University of Wisconsin--Madison\\
  1101 University Ave., Madison, Wisconsin 53706
\end{center}

\vspace{\fill}

*Corresponding Author \\
\hspace*{2ex} email: blaise.thompson@wisc.edu \\
\hspace*{2ex} phone: (608) 263-2573

\pagebreak
\setcounter{page}{1}
\pagenumbering{arabic}
\renewcommand{\thepage}{S\arabic{page}}
\renewcommand{\thefigure}{S\arabic{figure}}

\pagebreak
\renewcommand{\baselinestretch}{0.75}\normalsize
\tableofcontents
\renewcommand{\baselinestretch}{1.0}\normalsize

\section{Orchestration Examples}

Here we elaborate with detailed information about what orchestration looks like for \yaq{}.
As we discussed in Section 3 of the manuscript, \yaq{} is designed to accommodate a wide variety of orchestration tools and approaches.

\subsection{\texttt{busy} for ensuring order of operations}

Each message call over the \yaq{} interface returns rapidly, ensuring that client applications are not blocked.
Some messages initiate long-running operations, \emph{e.g.} motor motion and sensor measurement.
Separate messages are provided to retrieve results from \emph{e.g.} sensor measurement.

In order to know how long to wait, all \yaq{} daemons provide a message called “is\_busy”, which returns "true" while the long running action is not complete, and "false" once it is finished.
Additionally, multiple clients can communicate with the same daemon simultaneously. A complex instrument may involve multiple operators watching sensor data in real time, while one program is orchestrating the hardware and recording the data.

Script writers may simply wait for ``busy'' to return ``false''.

\begin{codefragment}{Python}
import yaqc
import time
motor = yaqc.Client(port=38000)
motor.set_position(500)  # initiate a long motor motion
while motor.busy():
    time.sleep(0.001)
print("done")  # will only print when motion is complete
\end{codefragment}

This same pattern exists for sensors and more complex peripherals.


We believe that Python and the Scientific Python ecosystem is a powerfully simple tool for defining experimental procedures.
Lightweight scripts such as these are encouraged by the \yaq{} project.
The following two sections contain example orchestration scripts for common tasks.

\clearpage
\subsection{Raster Script}

For those unfamiliar with \yaq{}, the best orchestration example is a simple self-contained script.

\begin{codefragment}{Python}
import time
import yaqc
motor1 = yaqc.Client(port=38000)
motor2 = yaqc.Client(port=38001)
sensor = yaqc.Client(port=38002)
data = []
for m1 in range(-10, 10, 1):
    motor1.set_position(m1)
    for m2 in range(0, 300, 5):
        motor2.set_position(m2)
        for c in [motor1, motor2]:
            while c.busy():
                time.sleep(0.001)
        sensor.measure()
        while sensor.busy():
            time.sleep(0.001)
        reading = dict()
        reading["timestamp"] = time.time()
        reading["motor1"] = motor1.get_position()
        reading["motor2"] = motor2.get_position()
        reading.update(sensor.get_measured())
        data.append(reading)
\end{codefragment}

Here we are doing a two dimensional motor scan while collecting data from a sensor.
Motor one is stepping from -10 to 10, and motor two is scanning from 0 to 300.
Two examples of polling while not busy are used: the first to ensure that motors have stopped moving before sensors acquire, and the second to ensure that sensors stop collecting before the next motor motion happens.
This ``tick tock'' moving multiple motors, collecting sensor data, and repeating is a common pattern.

\clearpage

\subsection{Non-Blocking Script}

Sometimes you want to acquire data while a peripheral continuously moves, asynchronously with data acquisition.
The following is an example where sample temperature is continuously ramped and data is collected every second.

\begin{codefragment}{Python}
import time
import yaqc
temp = yaqc.Client(port=38003)
sensor = yaqc.Client(port=38004)
data = []
# begin heating
temp.set_position(100)
# take data until temp greater than 90 deg
while temp.get_position() <= 90:
    reading = dict()
    reading["timestamp"] = time.time()
    reading["temperature"] = temp.get_position()
    # measure
    sensor.measure()
    while sensor.busy():
        time.sleep(0.001)
    reading.update(sensor.get_measured())
    data.append(reading)
    # wait for next second
    while time.time() - reading["timestamp"] < 1:
        time.sleep(0.001)
# stop heating
temp.set_position(25)
\end{codefragment}

Imagine that the temperature starts out at room temperature, 25.
We first set the temp client to 100, beginning to heat the sample.
The outer while loop will continue until the temperature reaches our desired end-point of 90.
Each second, the while loop collects the current temperature and the current sensor reading.
This type of experiment is common in the chemical sciences.

Note that we pay no attention to the busy state of the temp client.
When compared with the previous raster scanning example, we see that \yaq{} can be flexible about which hardware operations are kept synchronous and which are allowed to run ``freely'' without blocking other operations.
The ubiquity and simplicity of the busy message allows for this flexibility.

\clearpage

\subsection{Landis}

We discussed the Landis Group and their WiQK reactor in Section 3 of the manuscript.
The Landis Group has defined several procedures for their flow reactor.
These procedures are highly idiosyncratic and are based on the particular tubing lengths and the flow topology of the reactor.
Each procedure involves articulating solenoid valves while injecting and withdrawing syringe pumps with a particular timing.
The Landis Group has written Python functions that drive this hardware using \yaq{}.
The functions parameterize the procedures in terms of chemically interesting variables like flow rate
and reaction time.

WiQK procedures and nascent graphical user interface can be found on GitHub at \\ \url{https://github.com/uw-madison-chem-shops/yaqc-wiqk}.

\subsection{Wright \& Bluesky}

The Wright Group uses a purpose-built graphical program to perform data acquisition.
All of the hardware used is controlled via \yaq{}.
Data is collected using Bluesky \cite{AllanDanielB2019a}, which runs in a background process.

The graphical program includes a queue of acquisition procedures (plans, in Bluesky terminology).
In the main body of the queue window (Figure \ref{acq:fig:queue_tab}), a table represents the current state of the queue and history.
More recently enqueued items appear towards the top of the table, older items towards the bottom.
The items waiting in queue, thus, appear at the top, and the items in the history appear at the bottom.
The currently running item (if it exists) appears at the boundary between the queue and the history.
The enqueued and running items have descriptions shown in bright white, while the completed items appear with gray descriptions.
The right most column indicates the position in the queue, given as a zero-based index.
Items can be reordered by editing the value in that column.
On left, there are two columns with buttons: remove and load.
Items in the queue (but not the history) can be removed individually with the red ``REMOVE'' buttons.
All items can be loaded into the sidebar where the parameters can be modified and a similar (or identical) item enqueued.
The other information provided include the plan name, the status (enqueued, RUNNING, completed, failed, aborted, stopped or unknown) and the description field, which lists the arguments passed to the plan.
Hovering over the description shows the full JSON of the item, including an error message if applicable.

Along the left side of the window, there is a panel that allows enqueuing new items.
Figure \ref{acq:fig:grid_scan} shows the configuration panel for the most popular acquisition procedure, \texttt{grid\_scan}.
This plan allows an arbitrary number of dimensions to be scanned against one another, selecting the hardware to be scanned and the detectors to read.

The plot tab (Figure \ref{acq:fig:plot_tab}) presents a live representation of recent data collected for the current plan.
The primary window is a PyQtGraph \cite{pyqtgraph} plot showing the most recent five slices of collected data.
The current slice is the brightest cyan slice, with the previous slices shown as duller colors progressively until the last one which is gray.
At the top, the currently plotted channel name and most recent collected value are shown in large font size.
The plotting does handle array detectors, for which the top number shown is the maximum value of the most recent array collected.
At present, cameras and other higher dimensional sensors are not able to be plotted using the live plot tab.
On the right hand side, there are controls to adjust the plot.
At the top, a selector allows for choosing which channel to plot.
The second selector determines which values appear along the X-axis of the graph.
Additionally, there is a selector which allows for changing the units of the X-axis.
Finally, there is a Scan Index indicator, which indicates which indexed pixel was the most recent collected.

\clearpage

\begin{landscape}
\begin{figure}
\includegraphics[width=9in]{"figures/queue_tab.png"}
	\caption{The main queue window of \texttt{bluesky-cmds}.}
\label{acq:fig:queue_tab}
\end{figure}
\end{landscape}

\clearpage

\begin{figure}
\includegraphics[width=3in]{"figures/grid_scan.png"}
\caption{The user interface for enqueueing a grid scan plan.}
\label{acq:fig:grid_scan}
\end{figure}

\clearpage

\begin{landscape}
\begin{figure}
\includegraphics[width=9in]{"figures/plot_tab.png"}
	\caption{\texttt{bluesky-cmds} with the Plot tab open.}
\label{acq:fig:plot_tab}
\end{figure}
\end{landscape}

\clearpage


\section{\yaq{} transport layer}

\subsection{History of \yaq{} protocol format}

In the first days of \yaq{}, the message format was entirely made up of JSON strings, forming a generic Remote Procedure Call (RPC) framework.
There was no pre-formed protocol, though there were standardized messages to query for all allowed message names, signatures, and documentation.

Only after having a largely functional implementation, we descovered a community standard, JSON-RPC \cite{jsonrpc}.
While the exact format was not identical, almost all of the ideas we had built into our home baked JSON-based RPC were easy to translate into this standard format.
Instead of relying on our own esoteric format, we converted all \yaq{} daemons and clients to use JSON-RPC.

As \yaq{} gained hardware support and usage, it became evident that JSON was not an appropriate serialization format, especially when it comes to large binary data like images from cameras.
As such, we transitioned to using an RPC based on JSON-RPC, but using Msgpack \cite{msgpack} for serialization.
Msgpack is a format that can encode a strict superset of JSON messages, so swapping existing daemons required only changes to the core implementation and the client program (i.e. no changes to individual daemon implementations).
Along with using Msgpack, we defined an extension type to represent NDArrays.

The last major change to \yaq{}'s protocol format was the adoption of Apache Avro-RPC \cite{AvroSpecification}.
Avro is similarly terse like Msgpack for serializing messages, while having a well defined specification for defining an RPC schema: a static definition of available messages, including their signatures and documentation.
This statically defined schema pairs well with \yaq{}'s trait system.
Daemon authors provide a short, minimal TOML file which describes the implemented traits and unique messages for a particular daemon.
This then gets transformed by an automated tool to a full Avro-RPC JSON file describing the protocol.
Avro is then used to encode individual messages with minimal overhead.


\subsection{Scaling of large messages}

While most messages are intended to be short and quick to respond, large single messages will take appreciable time to transfer data from daemon to client over TCP.
One common usecase where a user may wish to transfer a large message over TCP would be camera data.
Cameras can have large arrays which contain the scientifically interesting data.

While \yaq{} is flexible enough to represent such arrays, it was not specifically designed with large cameras as the primary usecase.
As such, performance does suffer above about 1 Megapixel.
Figure \ref{si:fig:scaling} shows the relationship between number of pixels and time for a \yaq{} message to read the array.
Below about $2^{18}$ ($\approx250,000$) pixels, there is virtually no dependence on size, with the standard overhead of a \yaq{} message dominating the time for transfer.
\yaq{} remains usable for up to approximately $2^{20}$ ($\approx1,000,000$) pixels.
``Usable'' is a relative term which will depend greatly on context.
Here we generally mean ``seems responsive to a user trying to have feedback on human timescales''.
Applications which require high speed, high throughput cameras are unlikely to be suitable for \yaq{} even with relatively small cameras.
This test was conducted using 32-bit integer arrays.

\begin{figure}
  \includegraphics[width=.75\columnwidth]{figures/scaling/scaling}
	\caption{\label{si:fig:scaling} Scaling of transport time as a function of number of pixels for 32-bit integer arrays. Both the x- and y-axes are logarithmically scaled. Common camera sizes (512, 512), (1024, 1024), and (1024, 4096) are highlighted to provide context for camera users.}
\end{figure}


There are some strategies that could be used to mitigate the performance problems of large cameras, but they have not been implemented because large arrays remain an edge case among current \yaq{} users.
Such strategies could include enabling compression, using transport layers other than TCP, using regions-of-interest (ROIs) to limit array sizes, and writing arrays locally to disk and providing mechanisms for retrieval out of band.

The Avro Specification \cite{AvroSpecification} specifically provides for compression codecs in the case of storing to disk, but there is nothing preventing the RPC pipe from similarly encoding the data, provided both ends of the communication support it.
EPICS provides a similar NDArray data type, which has individual image level compression configureable \cite{normative_types}.
Compression would allow larger arrays to be transported using fewer bytes over TCP, which would likely improve performance for large arrays where TCP transport is the bottleneck.
While TCP is the only currently supported transport layer for \yaq{}, this limitation could be lifted in the future.

TCP was chosen as the preferred starting point because it is ubiquitous, being available on every desired platform and implementation language.
One alternative that would be relatively easy to support would be Unix domain sockets (UDS).
UDS has an extremely similar interface to TCP sockets, and are handled in Python by the same standard library module.
UDS is potentially more performant, but as the name suggests are limited to Unix-like systems (i.e. it will not work on Windows).

ROIs are implemented on the handful of cameras currently in the \yaq{} ecosystem \cite{yaqd_andor,yaqd_pi}, however we are still in the experimentation phase and have not standardized ROI configuration into a \yaq{} trait.
When only a subset of the total camera area is interesting, this is one way to limit array size over the interface.

Finally \yaq{} daemons could be implemented to write arrays locally to disk, rather than transferring over the \yaq{} interface directly.
Anecdotally, many users of Bluesky, EPICS, and related technologies use a similar strategy for large image data.
This has not yet been implemented for \yaq{}, and would require clients to have knowledge of how to retrieve and display the images collected, but there is nothing preventing this if throughput is a limitation.
If such behavior became a standard feature desired for cameras, it should be encapsulated in a \yaq{} trait to provide a consistent interface.

To reproduce this figure, you will need the libraries specified in ``figures/requirements.txt''.
The scripts for generating this figure, including the \yaq{} daemon, are located in ``figures/scaling''.
To generate the data, run the \yaq{} daemon using \texttt{python scaling.py --config config.toml} from with in the ``scaling'' folder.
Then run \texttt{python scaling-data-gen.py}, which will communicate with the running daemon and produce a CSV written to standard out.
The result can be saved to \texttt{scaling.csv}.
To visualize the results, run \texttt{python plot-scaling.py}, which will read the CSV and produce the image.

\clearpage

\section{\texttt{yaqd-control}: tooling for managing \yaq{} daemons}

\texttt{yaqd-control} is a command line utility provided by the \yaq{} project to manage \yaq{} daemons.
The primary purpose of the tool is to maintain a list of active daemons, provide access to their configuration, and their running state.
This includes querying the active state of known daemons, as well as starting daemons running as backgound services.

\subsection{installation}

yaqd-control can be installed via
PyPI \cite{yaqd-control} or
conda-forge \cite{yaqd-control-conda}.

\begin{codefragment}{bash}
$ pip install yaqd-control
\end{codefragment}

\begin{codefragment}{bash}
$ conda config --add channels conda-forge
$ conda install yaqd-control
\end{codefragment}

\subsection{Usage}

yaqd-control is a command line application.

Help: learn more, right from your terminal.

\begin{codefragment}{bash}
$ yaqd --help
Usage: yaqd [OPTIONS] COMMAND [ARGS]...

Options:
  --help  Show this message and exit.

Commands:
  clear-cache
  disable
  edit-config
  enable
  list
  reload
  restart
  scan
  start
  status
  stop
\end{codefragment}

Try \texttt{yaqd\ \ -\/-help} to learn more about a particular command.

\paragraph{the cache}\label{the-cache}

yaqd-control keeps track of known daemons, referred to as the cache

Status: yaqd-control can quickly show you the status of all daemons in
yaqd-control's cache. This is usually the most used subcommand, as it
gives a quick overview of the system, which daemons are offline, and
which are currently busy.

\begin{codefragment}{bash}
$ yaqd status
+-----------+-------+--------------------------+------+---------+-------+
| host      | port  | kind                     | name | status  | busy  |
+-----------+-------+--------------------------+------+---------+-------+
| 127.0.0.1 | 38202 | system-monitor           | foo  | online  | False |
| 127.0.0.1 | 39054 | fake-continuous-hardware | bar  | online  | True  |
| 127.0.0.1 | 39055 | fake-continuous-hardware | baz  | online  | False |
| 127.0.0.1 | 39056 | fake-continuous-hardware | spam | offline | ?     |
| 127.0.0.1 | 37067 | fake-discrete-hardware   | ham  | online  | False |
| 127.0.0.1 | 37066 | fake-discrete-hardware   | eggs | online  | False |
+-----------+-------+--------------------------+------+---------+-------+
\end{codefragment}

List: this is essentially the same as \texttt{status} except that it
does not attempt to contact the daemons, so it does not give you
additional context. List supports a flag \texttt{--format} which accepts "json",
"toml", "prettytable", or "happi".

\begin{codefragment}{bash}
$ yaqd list
+-----------+-------+--------------------------+------+
| host      | port  | kind                     | name |
+-----------+-------+--------------------------+------+
| 127.0.0.1 | 38202 | system-monitor           | foo  |
| 127.0.0.1 | 39054 | fake-continuous-hardware | bar  |
| 127.0.0.1 | 39055 | fake-continuous-hardware | baz  |
| 127.0.0.1 | 39056 | fake-continuous-hardware | spam |
| 127.0.0.1 | 37067 | fake-discrete-hardware   | ham  |
| 127.0.0.1 | 37066 | fake-discrete-hardware   | eggs |
+-----------+-------+--------------------------+------+
\end{codefragment}

Scan: Scanning allows you to add currently running daemons to the cache.

\begin{codefragment}{bash}
$ yaqd scan
scanning host 127.0.0.1 from 36000 to 39999...
...saw unchanged daemon fake-discrete-hardware:eggs on port 37066
...saw unchanged daemon fake-discrete-hardware:ham on port 37067
...found new daemon system-monitor:foo on port 38202
...found new daemon fake-continuous-hardware:bar on port 39054
...saw unchanged daemon fake-continuous-hardware:baz on port 39055
...known daemon fake-continuous-hardware:spam on port 39056 not responding
...done!
\end{codefragment}

Scan has some additional options, passed as flags on the command line,
which allow you to change the default scan range and host (for remotely
accessed daemons):

\begin{codefragment}{bash}
$ yaqd scan --help
Usage: yaqd scan [OPTIONS]

Options:
  --host TEXT      Host to scan.
  --start INTEGER  Scan starting point.
  --stop INTEGER   Scan stopping point.
  --help           Show this message and exit.
\end{codefragment}

Edit Config: yaqd-control provides an easy way to edit the default
config file location for a daemon kind. This uses your default editor
(\texttt{EDITOR} environment variable), and defaults to \texttt{notepad.exe} on
Windows, and \texttt{vi} on other platforms. Using yaqd-control to edit
config files means that you do not need to know the default location.
Additionally, it does some basic validity checks (that the toml parses
and that each daemon section has the \texttt{port} keyword). If an error
is found, you are prompted to re-edit the file. Daemons from the config
file are added to the cache. You may pass multiple daemon kinds, which
will be opened in succession.

\begin{codefragment}{bash}
$ yaqd edit-config fake-continuous-hardware system-monitor
\end{codefragment}

Clear Cache: Note that this is a destructive action.
\texttt{clear-cache} deletes all daemons from the cache (thus
\texttt{list} and \texttt{status} will give empty tables) There is no
user feedback.

\begin{codefragment}{bash}
$ yaqd clear-cache
$ yaqd status
+------+------+------+------+--------+------+
| host | port | kind | name | status | busy |
+------+------+------+------+--------+------+
+------+------+------+------+--------+------+
\end{codefragment}

\paragraph{Running in the background}\label{running-in-the-background}
Each of the commands in this section can take multiple daemon kinds.
Enable: by enabling a daemon, you allow the operating system to manage
that daemon in the background. An enabled daemon will always start again
when you restart your computer. Enabling is required for the rest of the
commands in this section to work as expected. After enabling, it's
typical to start the daemon as well, this does not happen automatically.
Enablement works in slightly different ways on different platforms, but
the commands are the same (don't worry if the password prompts are
different). Currently supported platforms are Linux (systemd), MacOS
(launchd) and Windows (via NSSM, bundled with the distribution).

\begin{codefragment}{bash}
$ yaqd enable system-monitor
[sudo] password for scipy2020:
==== AUTHENTICATING FOR org.freedesktop.systemd1.manage-unit-files ===
Authentication is required to manage system service or unit files.
Password:
==== AUTHENTICATION COMPLETE ===
\end{codefragment}

Disable: this is the inverse operation to enable, which makes it so that
the daemon does not start on reboot. This does not affect the running
daemon.

\begin{codefragment}{bash}
$ yaqd disable system-monitor
==== AUTHENTICATING FOR org.freedesktop.systemd1.manage-unit-files ===
Authentication is required to manage system service or unit files.
Password:
==== AUTHENTICATION COMPLETE ===
Removed /etc/systemd/system/multi-user.target.wants/yaqd-system-monitor.service.
\end{codefragment}

Start: This starts the daemon running in the background immediately. It
must have been enabled to run in the background using this command.

\begin{codefragment}{bash}
$ yaqd start system-monitor
==== AUTHENTICATING FOR org.freedesktop.systemd1.manage-units ===
Authentication is required to start 'yaqd-system-monitor.service'.
Password:
==== AUTHENTICATION COMPLETE ===
\end{codefragment}

Stop: This stops the daemon running in the background immediately. It
must have been running in the background using yaqd-control (either on
startup via enable or via the start command above).

\begin{codefragment}{bash}
$ yaqd stop system-monitor
==== AUTHENTICATING FOR org.freedesktop.systemd1.manage-units ===
Authentication is required to stop 'yaqd-system-monitor.service'.
Password:
==== AUTHENTICATION COMPLETE ===
\end{codefragment}

Restart/Reload: This stops (if running) and restarts the daemon running
in the background immediately. Reload is slightly different in that it
signals to the daemon to reload its configuration rather than completely
restart, but effectively it is the same as restart (and is a pure alias
where such a signal is not supported). It must have been enabled to run
in the background using this command.

\begin{codefragment}{bash}
$ yaqd restart system-monitor
==== AUTHENTICATING FOR org.freedesktop.systemd1.manage-units ===
Authentication is required to restart 'yaqd-system-monitor.service'.
Password:
==== AUTHENTICATION COMPLETE ===
\end{codefragment}


\clearpage

\section{Package size analysis}

The raw data for package size analysis was collected using Tokei \cite{tokei}, a command line tool for analysing line lengths of source code.
It includes breaking down line length into ``code'', ``comments'', and ``blank'' lines.
The package size data was curated into \texttt{figures/lines\_histogram.txt}, which includes annotations as to the ``type'' and ``class'' for each file.
The ``type'' annotation is one of ``stub'' (meaning that the bulk of the implementation is in another file), ``normal'' (an individual daemon with the bulk of its implementation), ``protocol'' (an implementation the provides for multiple daemons, such as those referred to by ``stub''s, or ``serial'' (the implementation of a serial protocol rather than a full fledged daemon).
The ``class'' refers to the primary function of the daemon and is one of the following values: ``is-sensor'' for detectors, ``has-position'' for setable hardware, ``serial'' (identical to ``type'' annotation), and ``other'' for daemons which do not fit in the above categories.

The plotting script \texttt{figures/lines\_histogram.py} can be used to generate the figure from this text file.
The figure uses the ``code'' lines to generate a histogram of file sizes in the \yaq{} python implementations.
This is a measurable proxy for the amount of work that implementing an individual \yaq{} daemon entails, though of course cannot capture the work involved in learning the lower level interface.

\section{The name \yaq{}}

The name \yaq{} started out as an ``inside joke'' with several meanings. \\
For posterity, we describe those meanings here.
\begin{enumerate}
  \item \yaq{} is similar to the popular acronym ``daq'' (Data AcQuisition)
  \item \yaq{} could stand for Yet Another acQuisition, a reference to the popular tongue-in-cheek naming convention ``yet another X'' in software development \cite{Raymond_1996}. This is a particularly poignant reference for the authors, who are well aware that \yaq{} is ``yet another project'' in a long line of increasingly sophisticated hardware interface software packages in the Wright Group.
  \item \yaq{} can reference the animal ``yak'', \textit{Bos grunniens}, who serves as a mascot for the project.
  \item \yaq{} is an obscure reference to the term ``yak shaving'', coined by Carlin Vieri at MIT in the 1990s \cite{CarlinVieri}. Quoting from Wiktionary \cite{yak_shaving}: yak shaving is \textit{any apparently useless activity which, by allowing you to overcome intermediate difficulties, allows you to solve a larger problem}. The entire \yaq{} project is one giant ``yak shaving'' task that ultimately enables progress towards our primary mission---creative, reproducible, insightful experimental science.
\end{enumerate}

\clearpage

\printbibliography

\end{document}
